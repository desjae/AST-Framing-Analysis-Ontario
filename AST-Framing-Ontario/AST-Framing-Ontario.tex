\documentclass[]{elsarticle} %review=doublespace preprint=single 5p=2 column
%%% Begin My package additions %%%%%%%%%%%%%%%%%%%
\usepackage[hyphens]{url}

  \journal{Social Science \& Medicine} % Sets Journal name


\usepackage{lineno} % add
\providecommand{\tightlist}{%
  \setlength{\itemsep}{0pt}\setlength{\parskip}{0pt}}

\usepackage{graphicx}
%%%%%%%%%%%%%%%% end my additions to header

\usepackage[T1]{fontenc}
\usepackage{lmodern}
\usepackage{amssymb,amsmath}
\usepackage{ifxetex,ifluatex}
\usepackage{fixltx2e} % provides \textsubscript
% use upquote if available, for straight quotes in verbatim environments
\IfFileExists{upquote.sty}{\usepackage{upquote}}{}
\ifnum 0\ifxetex 1\fi\ifluatex 1\fi=0 % if pdftex
  \usepackage[utf8]{inputenc}
\else % if luatex or xelatex
  \usepackage{fontspec}
  \ifxetex
    \usepackage{xltxtra,xunicode}
  \fi
  \defaultfontfeatures{Mapping=tex-text,Scale=MatchLowercase}
  \newcommand{\euro}{€}
\fi
% use microtype if available
\IfFileExists{microtype.sty}{\usepackage{microtype}}{}
\bibliographystyle{elsarticle-harv}
\usepackage{graphicx}
\ifxetex
  \usepackage[setpagesize=false, % page size defined by xetex
              unicode=false, % unicode breaks when used with xetex
              xetex]{hyperref}
\else
  \usepackage[unicode=true]{hyperref}
\fi
\hypersetup{breaklinks=true,
            bookmarks=true,
            pdfauthor={},
            pdftitle={The framing of active school travel in Ontario, Canada as a health and built environment issue},
            colorlinks=false,
            urlcolor=blue,
            linkcolor=magenta,
            pdfborder={0 0 0}}
\urlstyle{same}  % don't use monospace font for urls

\setcounter{secnumdepth}{0}
% Pandoc toggle for numbering sections (defaults to be off)
\setcounter{secnumdepth}{0}

% Pandoc citation processing

% Pandoc header
\usepackage{booktabs}
\usepackage{longtable}
\usepackage{array}
\usepackage{multirow}
\usepackage{wrapfig}
\usepackage{float}
\usepackage{colortbl}
\usepackage{pdflscape}
\usepackage{tabu}
\usepackage{threeparttable}
\usepackage{threeparttablex}
\usepackage[normalem]{ulem}
\usepackage{makecell}
\usepackage{xcolor}



\begin{document}
\begin{frontmatter}

  \title{The framing of active school travel in Ontario, Canada as a health and
built environment issue}
    \author[Some Department]{Author 1\corref{1}}
   \ead{author1@example.com} 
    \author[Some Department]{Author 2}
   \ead{author2@example.com} 
    \author[Another University]{Author 3\corref{2}}
   \ead{author3@example.com} 
    \author[Some Institute]{Author 4\corref{2}}
   \ead{author4@example.com} 
    \author[Some University]{Author 5}
   \ead{author5@example.com} 
    \author[Some Department]{Author 6}
   \ead{author6@example.com} 
      \address[Some Department]{Department, Street, City, Province, Postal Code}
    \address[Another University]{Department, Street, City, Province, Postal Code}
    \address[Some Institute]{Street, City, Province, Postal Code}
    \address[Some University]{Department, Street, City, Province, Postal Code}
      \cortext[1]{Corresponding Author}
    \cortext[2]{Equal contribution}
  
  \begin{abstract}
  This is the abstract.
  
  It consists of two paragraphs.
  \end{abstract}
  
 \end{frontmatter}

\textbf{\emph{Background}}:\\
\textbf{\emph{Methods}}:\\
\textbf{\emph{Results}}: \textbf{\emph{Conclusions}}:

\newpage

\hypertarget{introduction}{%
\section{1. Introduction}\label{introduction}}

Rates of walking and bicycling to school, commonly known as active
school travel (AST), have been declining in Canada and the United States
for decades (Rothman et al., 2018), with levels much lower than other
developed countries like The Netherlands (van Goeverden and de Boer,
2013), Denmark (Jensen, 2008), and Japan (Waygood and Taniguchi, 2020).
However, the appetite for AST may be high in many Canadian communities.
For instance, 40\% of children in a study conducted in Toronto, Ontario
who are driven to school would like to travel by bicycle instead
(Larouche et al., 2016). But the study reported that less than 3\%
actually do, despite the vast majority having access to a bicycle and
living within a short and bikeable distance to school. Another study
based in London, Ontario reported similar findings with respect to
children's preference for active travel (Larsen et al., 2012). From a
public health perspective, this represents a huge opportunity to improve
health and wellbeing if more children could use active modes to school.

In response to this trend, the Government of Ontario, in Canada, created
a fund in 2017 to support communities across the province in developing
AST initiatives. Green Communities Canada runs the program, called
\emph{Ontario Active School Travel}, and determines the projects that
will be funded with their support (Canada, 2020a). As of December 2021,
OAST has awarded over 2 million Canadian dollars and provided resources
for over 25 projects across Ontario. Many communities implemented school
travel planning (STP), along with encouragement activities like walking
school buses or developed resources for schools and parents, among other
actions (see Canada, 2020b). In this context, STP is a popular
``school-specific'' intervention led by a facilitator that brings
together a committee of stakeholders from diverse sectors including
education, planning, transportation, and public health to develop action
plans (Buliung et al., 2011; Mammen, M. R. Stone, Ron Buliung, et al.,
2014). The five-step process involves identifying barriers to AST based
on the local context and implementing approaches or activities following
the 5 E's that alleviate concerns about AST and make AST safer and more
convenient (Buliung et al., 2011; Lang et al., 2011).

How AST is framed by STP stakeholders to a target audience, like parents
or the general public, may raise awareness about the issue and influence
how walking and bicycling to school are perceived. Gamson and Modigliani
(1987) define a frame as a ``central organizing idea or story line that
provides meaning'' to a particular phenomenon. A frame can enable
individuals ``to locate, perceive, identify, and label'' information
pertaining to various dimensions of an issue (Goffman, 1974). Therefore,
stakeholder groups involved in STP efforts can play a role in shaping
public perception about AST in such a way that it attracts greater
attention and becomes a ``problem'' that needs to be addressed through
behaviour change or by new policies. We also contend that the ways in
which the AST issue is framed can affect the policy process, as outlined
in Kingdon's (1984) Multiple Streams Framework (MSF), through all three
streams (i.e., problem, policy, and politics). This framing can
ultimately impact financial contributions and potential solutions to
address the decline of AST in Canada.

After significant investment of human and financial resources to boost
rates of AST in Ontario over the past few years, we ask the following
questions: How is AST framed? What benefits of AST are communicated to
the public? What solutions are proposed to increase rates of AST? Do
proposed solutions focus primarily on the intrapersonal and
interpersonal factors, meaning that behaviour change must ultimately
result from the individual or household making different travel
decisions? Or is AST framed as an issue that must be addressed through
changes in the built environment or at the policy level?

In this paper, we use text mining and topic modelling to examine how
three STP stakeholder groups in Ontario (i.e., municipalities, school
boards, and transportation consortia) frame the issue of AST. We
assembled a corpus of texts from webpages that can be considered an
important source of information for the general public or parents who
are interested in AST. We examine word frequency, bigrams, and
concordances in these selected documents, and also identify key topics
presented by each stakeholder group. We then compare the findings from
these documents to a selection of studies on AST and explore the extent
to which there is concordance between the literature on AST and
materials shared with the public.

\hypertarget{literature-review}{%
\section{2. Literature Review}\label{literature-review}}

\hypertarget{benefits-of-active-school-travel}{%
\subsection{2.1. Benefits of active school
travel}\label{benefits-of-active-school-travel}}

The benefits of AST are valuable information to communicate to the
public in order to convey the importance of this issue. The desire to
increase AST in Canada is certainly warranted - there is compelling
evidence that children who actively commute to school earn physical and
mental health benefits. Faulkner et al.~(2009) concluded from their
systematic review that children who travel on foot or by bicycle to
school generally have higher levels of physical activity than their
peers who are driven to school. However, Schoeppe et al.~(2015) reported
no association between AST and physical activity. This relationship
could be dose-dependent, meaning that children would have to travel
longer distances to accumulate physical activity. A walking distance of
1000-1600 metres to school has been found to contribute to overall
levels of physical activity for boys (Faulkner et al., 2013). The daily
routine of travelling to school can be a good opportunity for children
to regularly build physical activity into their schedule (Mitra, 2013).
Research has also shown that using active modes to school contributes to
improvements in cardiovascular fitness (Børrestad et al., 2012).

More recently, there has been literature exploring the link between
transport and children's wellbeing (Waygood et al., 2020), with relevant
applications to the study of travel satisfaction (van den Berg et al.,
2020; Westman, Olsson, et al., 2017). Being driven to school reduces
community interactions for children which may impact their social
wellbeing (Waygood and Friman, 2015), whereas walking and bicycling give
children opportunities to socialize with friends or siblings (Michail et
al., 2021). This is something that children highly value about travel
(Zwerts et al., 2010) and indicates that social connections through
travel are important for their wellbeing. AST also provides
opportunities for children to engage with the natural environment (Fusco
et al., 2012; Romero, 2015) which can improve their mental or emotional
wellbeing.

\hypertarget{factors-that-influence-active-school-travel-and-mode-choice}{%
\subsection{2.2. Factors that influence active school travel and mode
choice}\label{factors-that-influence-active-school-travel-and-mode-choice}}

Factors that influence AST have been presented and organized using a
socio-ecological model (SEM) (Mitra, 2013) or systems model (Badland et
al., 2016) whereby children's travel behaviour is understood within the
context of their household, social, neighbourhood, and policy
environments. Individual characteristics of the child are also
determinants. The SEM comes from the field of public health and is a
useful framework for understanding complex health behaviours, including
travel habits such as walking or bicycling to school, because it
identifies multiple determinants that need to be addressed by
interventions to facilitate behaviour change. The consensus in the
literature is that various factors are interrelated and that
interventions should target multiple levels in order to increase levels
of healthy and physically active modes of travel to school (Mitra,
2013). Rather than describing an extensive list of factors that
influence AST and mode choice in Canada, which has been covered
elsewhere (e.g., Mammen et al., 2012; Mitra, 2013; Rothman et al., 2018;
Wilson et al., 2018), we briefly discuss a few potentially modifiable
determinants that may be targeted for change through STP.

At the individual level, older child age is often associated with AST
(Mammen et al., 2012; Stark et al., 2018; Wilson et al., 2018) as
parents typically escort younger children. There is evidence that gender
is a determinant of AST, with boys being more likely to travel using
active modes than girls, although this is not a strong or consistent
finding (Rothman et al., 2018; Schoeppe et al., 2015). Children's mode
choice to school is strongly influenced by their parents' travel
behaviours and the complexity of their household's travel needs (Buliung
et al., 2021), which indicates that shifting parental perceptions and
habits is important. Convenience and inclement weather have been cited
by parents as barriers to AST (Buliung et al., 2011). Parental
perceptions of the built or school environment (De Meester et al., 2014;
Panter et al., 2010) and their children's skills (Mammen et al., 2012)
also influence whether they allow their children to walk or bicycle to
school.

Distance between home and school is most strongly associated with AST
(Ikeda et al., 2018; Mammen et al., 2012; Pont et al., 2009; Rothman et
al., 2018) with less AST reported among children who have to travel
farther to school. Many studies have also found that the quality of the
built environment along the route to school and around the school site
(Ikeda et al., 2018; Rothman et al., 2021) and provision of active
travel infrastructure (Chen et al., 2018; Pont et al., 2009) facilitate
AST. Canadian youth report that they feel most safe bicycling on streets
in their neighbourhood or that have low volumes of traffic
({\textbf{???}}). Finally, concerns about traffic and strangers have
been reported by parents who drive their children to school (Mammen et
al., 2012), which highlights that the volume and speed of cars can be a
concern or deterrent for AST.

\hypertarget{school-travel-planning-in-canada}{%
\subsection{2.3. School travel planning in
Canada}\label{school-travel-planning-in-canada}}

School travel planning (STP) has been implemented in Canada since at
least the late 2000s. Within the STP process, facilitators generally
establish multi-sector committees who intervene at the participating
school through a range of activities related to the 5E's such as
\emph{education} strategies, \emph{encouragement} through in-person
events or programs, \emph{engineering} improvements to or around the
school site, and \emph{enforcement} of traffic speeds around schools
(Lang et al., 2011; Mammen, M. R. Stone, G. Faulkner, et al., 2014). The
first large-scale evaluation of STP as an intervention in Canada took
place at twelve schools across the country, including four in Ontario,
using parental surveys to measure changes in travel behaviour and
perceptions (Buliung et al., 2011). Since then, there have been other
assessments of both the efficacy of STP (Buttazzoni et al., 2019;
Mammen, M. R. Stone, G. Faulkner, et al., 2014) and the process of
implementing such programs in Ontario (Buttazzoni et al., 2018; Mammen
et al., 2015). STP facilitators have recommended that additional time
and resources are needed to improve the efficacy of STP (Mammen et al.,
2015), which highlights that long-term and sustained efforts driven at
the policy level are required to address declining rates of AST.

The ways in which AST is framed seem particularly important to shift
parental attitudes and perceptions given their reported influence on
children's travel mode to school. STP activities heavily focus on
education or encouragement (Buliung et al., 2011; Buttazzoni et al.,
2018; Mammen, M. R. Stone, G. Faulkner, et al., 2014), but parents may
not always be receptive to the goals of STP and may be resistant to
behaviour change (see Buttazzoni et al., 2018). Parents have been found
to express different understandings, language, and perceptions than
planners of the built environment determinants that influence school
travel (Buliung et al., 2021). This is also true when it comes to other
factors like convenience of different modes to school (Lang et al.,
2011). STP stakeholder groups must pay special attention to parents'
understanding of the decline of AST as a problem, which may affect their
receptivity to proposed solutions.

The ``central organizing idea or story line'' of AST, to apply the
definition from Gamson and Modigliani (1987), could also affect broader
support in the community. Municipal representatives are perceived to be
instrumental but the involvement of other stakeholder groups (e.g.,
busing consortium representatives and local residents) can be lacking
(Buttazzoni et al., 2018). In Ontario, it would be reasonable to say
that AST has become a policy issue on the education and public health
agendas over this time as evidenced by the financial contribution from
the provincial government. The support from a range of municipal
representatives (see Buttazzoni et al., 2018; Mammen et al., 2015)
demonstrates that the ``policy stream'' (see Kingdon, 1984) has been
well engaged. It is unknown to what degree the general public (i.e.,
local residents) has been exposed to messaging about AST, which could
affect their participation in the STP process as desired by STP
facilitators (Buttazzoni et al., 2018).

The success of STP interventions would likely depend on parental
judgments of factors that are related to school travel, as well as
support from key policy makers. For example, parents have been found to
view mixed land use as conducive for driving, despite transport planners
viewing neighbourhoods with mixed uses as key for encouraging more
active travel (Buliung et al., 2021). Therefore, stakeholder groups
involved in STP must make important choices about the proposed solutions
and potential benefits that ought to be communicated to parents and the
general public about AST to convey its importance as a policy issue and
to facilitate adoption of AST. Publicly available content about AST,
therefore, needs to effectively engage multiple audiences on this policy
issue including parents, children, politicians, and school
representatives. This information should reflect current knowledge from
research on school travel, plus content specific to local factors that
influence AST, so that the challenges of AST are adequately defined and
the opportunities or solutions to address the problems are clear.
However, it is unknown to what degree STP materials presented to the
general public reflect the scope of evidence from the literature on AST.

\hypertarget{data}{%
\section{3. Data}\label{data}}

\hypertarget{data-retrieval}{%
\subsection{3.1. Data retrieval}\label{data-retrieval}}

\hypertarget{policy-documents}{%
\subsubsection{3.1.1. Policy documents}\label{policy-documents}}

We assembled a collection of publicly available documents that were
sourced online from the main stakeholder groups involved in STP
initiatives in Ontario: i) school boards (public or Catholic and
English-speaking only); ii) municipal governments; and iii)
transportation consortia. The latter are a unique group of entities
sanctioned by Ontario's Ministry of Education in Ontario 2006. Each
consortium involves a collaboration between regions and school boards;
the consortium's objective is to deliver more efficient and timely
transportation services to schools in each region. Non-profit
organizations, police services, and advocacy groups are other
stakeholders who often play a role in supporting AST and/or STP, but
this study does not include any documents from these groups because they
are not consistently participating in all initiatives across Ontario.

The search was guided first by a list of all English public and Catholic
school boards across Ontario. The websites of each school board were
manually searched for pages related to school transport or travel. Any
pages relevant to these topics were manually downloaded. Next, we
collected documents by searching municipal government and transportation
consortia websites. These were identified based on geographic area
(i.e., the municipalities and/or transportation consortia that are in
the same geographic area of each school board). Webpages related to
active transport or school travel were manually downloaded.

Webpages from STP stakeholder groups were included in our analysis if
they were easy to find. This primary criterion was important since our
analysis pertains to how such issues are framed to the general public.
Thus, we included only webpages that were readily accessible, which we
defined as requiring no more than 2-4 separate links from the initial
Google search.

The initial corpus of documents from STP stakeholder groups included 69
relevant webpages (i.e., one page or more) from all STP stakeholder
groups. We refer to these as policy documents throughout the paper. It
is important to note that school boards, municipalities, and
transportation consortia may or may not publish information about their
involvement in AST and STP efforts on their respective websites or in
policy documents. Search results are summarized in Table
\ref{tab:policy-documents}.

\begin{table}

\caption{\label{tab:policy-documents}\label{tab:search-results}Search results from the main STP stakeholder groups.}
\centering
\resizebox{\linewidth}{!}{
\begin{tabular}[t]{>{}l|l|>{}l}
\toprule
Stakeholder & Total & Included\\
\midrule
\cellcolor{gray!6}{\textbf{School boards}} & \cellcolor{gray!6}{62} & \cellcolor{gray!6}{32}\\
\textbf{Municipalities} & 62 & 28\\
\cellcolor{gray!6}{\textbf{Transportation consortia}} & \cellcolor{gray!6}{39} & \cellcolor{gray!6}{9}\\
\bottomrule
\end{tabular}}
\end{table}

\hypertarget{academic-papers}{%
\subsubsection{3.1.2. Academic papers}\label{academic-papers}}

\textbf{To be completed by Dr.~Paez}

\hypertarget{data-cleaning}{%
\subsection{3.2. Data cleaning}\label{data-cleaning}}

A multi-step process was conducted to ensure that the analysis captured
as much text as possible from both the policy documents (n = 64) and
academic papers (n = 233). To begin, the webpages, which were manually
downloaded in portable document format (PDF), were trimmed so that pages
that only consisted of tables, figures, or references were removed. Many
academic papers were in a two-column format, which is not ideal for
conversion to \texttt{txt}. We adapted a procedure
(https://stackoverflow.com/questions/42541849/extract-text-from-two-column-pdf-with-r)
to read the two-column PDF documents so that they would be converted
correctly. Four academic papers did not join sufficiently and were taken
out of the corpus due to the substantial time required to manually
correct their inconsistencies.

Next, we converted the trimmed PDF documents into \texttt{txt} files so
that they could be imported in R for analysis. We then proceeded to a
manual cleaning phase where we removed any remaining tables, figures,
references, headers/footings, and captions that could not be trimmed.
Manual corrections were also required for certain pages in academic
papers that remained in two-column format after the conversion process.
This typically occurred on pages that had a table or figure which
disrupted the text. Finally, we reviewed all of the documents to remove
hyphenation by line breaks and to keep hyphenated words together on the
same line. Any ligatures (e.g., combinations of characters or letters
that were not properly detected during the conversion process) were
fixed by replacing the unicode sequence of character by inserting the
missing sequence of characters.

We also manually removed any extraneous material in the academic papers
that did not pertain to AST specifically. This included footnotes,
references, acknowledgments, and conflict of interest statements in the
academic papers. We removed all phone numbers, inserted links to other
webpages, personal names, and content not to specific to AST from the
policy documents that were retrieved from the websites of school boards,
municipalities, and transportation consortia.

In the final step, we removed all blank spaces, punctuation,
capitalization, and numbers. English stop words, which are common words
such as \emph{and} or \emph{the} as identified in a predetermined list
by Lewis et al.~({\textbf{???}}) and other frequent terms in the
documents like ``school'' and specific location names, were removed from
the corpora.

\hypertarget{methods}{%
\section{4. Methods}\label{methods}}

\hypertarget{framing-analysis}{%
\subsection{4.1. Framing analysis}\label{framing-analysis}}

Issues that pertain to public health or wellbeing are often presented to
the public through particular frames to influence perceptions or
behaviours. As previously mentioned, a frame is a ``central organizing
idea or story line that provides meaning'' to a public issue or
phenomenon (Gamson and Modigliani, 1987). Scholars in the field of
political communications have proposed that communicators, such as the
media or an institution, construct the narrative of a frame for policy
positions or public issues in order to activate or restrict a particular
response in the intended audience (Pan and Kosicki, 1993). Organized
groups of stakeholders can employ similar methods to attract attention
to particular issues. Framing can be used to position existing solutions
as suitable to address particular issues (Mah et al., 2014), which may
prevent the public from being aware of other policy approaches that
challenge the status quo. The way policy issues are framed is ultimately
important to understand because it plays a role in either altering or
preserving the existing social perceptions. This, in turn, can affect
whether issues are put on the agenda of policy makers and can determine
which solutions are proposed to address the problem (Kingdon, 1984).

Framing of issues is an important step in developing health policy. An
obvious example over the past decade is the framing of climate change as
a public health issue (e.g., Depoux et al., 2017; Maibach et al., 2010;
Weathers and Kendall, 2016) to increase public engagement and awareness
of the issue. This framing has slowly advanced this issue on public
policy agendas as public attention puts pressure on the policy stream to
adopt frameworks for action. For example, transport planners also use
different frames to guide the extent to which transport policies can be
adapted to address climate change. In a recent paper (Reynard et al.,
2021), framing analysis was applied to review the representation of
issues such as mobility and social exclusion in municipal policies from
four western Canadian cities under the current circumstances of climate
change. The authors found four primary frames: ``The Growing City'',
``If You Build It, They Will Come'', ``Better City for All'', and a
``the Resilient City'' (Reynard et al., 2021). Each frame presented the
nature, opportunities, and challenges of climate change in different
ways which set the stage for the types of mitigation and adaptation
strategies that cities were proposing to address this issue.

In a similar way, we hypothesize that STP stakeholder groups in Ontario
have framed AST in particular ways to effectively engage multiple
audiences on this policy issue including parents, local residents,
school representatives, and municipal representatives. These groups are
likely identifying points of intervention and potential opportunities to
build support for and encourage AST.

\hypertarget{topic-modelling}{%
\subsection{4.2. Topic modelling}\label{topic-modelling}}

We use topic modelling in R to conduct the framing analysis. Topic
modelling is a machine learning technique that can identify what
language and concepts are being communicated by analyzing text. This
method is more practical for researchers working with large amounts of
text because it replaces the manual coding of topics that would normally
take place to analyze or summarize textual data (Jacobi et al., 2016).
In the data pre-processing phase, we tokenize the text in the documents
and create a document-term matrix so that it is in the correct format
for analysis. We primarily use the following packages: \texttt{tidytext}
({\textbf{???}}), \texttt{topicmodels} ({\textbf{???}}),
\texttt{word2vec} ({\textbf{???}}), and \texttt{wordcloud}
({\textbf{???}}) to examine text in the documents that were sourced for
this project. These packages have functions for determining the
frequency of specific words in each document or relationships (e.g.,
pairs of adjacent terms called bigrams) and correlations between words.
These methods can reveal what language and concepts are being
communicated to the general public. Topic modelling is a popular method
for analyzing text from social media platforms (Albalawi et al., 2020)
or news articles (Jacobi et al., 2016). We estimate latent Dirichlet
allocation (LDA) models to classify both the STP and documents according
to the topics that are contained within them. This method ``treats each
document as a mixture of topics, and each topic as a mixture of words''
(Silge and Robinson, 2021). The model's output is ``a set of topics
consisting of clusters of words that co-occur in these documents
according to certain patterns'' (Jacobi et al., 2016). Researchers must
then interpret the identified topics, as done after other methods of
manual coding. We also compare the topics between the policy documents
and the academic papers, and use our interpretation to answer the
questions outlined above that guide this paper. This methodology is not
familiar to many, so we describe in more detail below how we used
specific functions and code in R to produce our results.

\hypertarget{reproducibility}{%
\subsection{4.3. Reproducibility}\label{reproducibility}}

This paper is an example of open and reproducible research that uses
only open software. All data were obtained from publicly available
sources and organized in the form of a data package. Following best
practices in spatial data science ({\textbf{???}}), the code and data
needed to reproduce or conduct a similar analysis for other regions in
North America or elsewhere are available for download.

\hypertarget{results}{%
\section{5. Results}\label{results}}

\hypertarget{word-and-document-frequency}{%
\subsection{5.1. Word and document
frequency}\label{word-and-document-frequency}}

We analyzed word and document frequency for each set of documents (i.e.,
corpus). Table \ref{tab:word-table} shows the most frequent terms found
in the municipal, transportation consortia, school board, and academic
documents. As expected, policy documents and academic papers reference
\emph{active travel}, \emph{walking}, \emph{biking} or \emph{cycling},
and \emph{students} more than other terms. Each corpus also has
\emph{safety} and \emph{traffic} as common words which suggests
congruence on these key factors between the research literature and how
AST is framed to the public. The word \emph{physical} is present in each
corpus, but it's not clear what this refers to (e.g, \emph{physical
activity}, \emph{physical health}, or the \emph{physical environment}).
Furthermore, documents from STP stakeholder groups discuss
\emph{resources}, \emph{information}, and \emph{services} about school
travel. In the section below, the context in which these terms are
discussed is explored further.

Unlike the academic papers, policy documents include the words
\emph{route} or \emph{routes} as frequent terms. This could reflect the
role of STP stakeholder groups in identifying safe routes to school to
share with parents or families, as well as the STP emphasis on making
the physical environment safer for AST. The academic corpus differs from
the policy documents in that \emph{parents} and \emph{distance} are the
second and third most common terms. In addition, \emph{time},
\emph{factors}, \emph{environment}, and \emph{age} are also identified
in the academic papers. These words are absent from the list of common
words in policy documents. Table \ref{tab:word-table} indicates that the
academic corpus discusses a broader range of determinants of AST than
the policy documents. The number of references for each term in the
academic papers is also significantly higher due to the inclusion of
more documents.

\begin{table}

\caption{\label{tab:word-table}\label{tab:word-table}Top 25 terms identified in each corpora. Document frequencies are also indicated.}
\centering
\resizebox{\linewidth}{!}{
\begin{tabular}[t]{lcclcclcclcc}
\toprule
\multicolumn{3}{c}{Municipalities} & \multicolumn{3}{c}{School Boards} & \multicolumn{3}{c}{Transportation Consortia} & \multicolumn{3}{c}{Academic Papers} \\
\cmidrule(l{3pt}r{3pt}){1-3} \cmidrule(l{3pt}r{3pt}){4-6} \cmidrule(l{3pt}r{3pt}){7-9} \cmidrule(l{3pt}r{3pt}){10-12}
Term & Count (n) & Documents (n) & Term & Count (n) & Documents (n) & Term & Count (n) & Documents (n) & Term & Count (n) & Documents (n)\\
\midrule
\cellcolor{gray!6}{active} & \cellcolor{gray!6}{248} & \cellcolor{gray!6}{26} & \cellcolor{gray!6}{active} & \cellcolor{gray!6}{124} & \cellcolor{gray!6}{13} & \cellcolor{gray!6}{active} & \cellcolor{gray!6}{67} & \cellcolor{gray!6}{7} & \cellcolor{gray!6}{walking} & \cellcolor{gray!6}{5137} & \cellcolor{gray!6}{222}\\
travel & 126 & 20 & bus & 120 & 20 & walking & 55 & 8 & parents & 3946 & 211\\
\cellcolor{gray!6}{walking} & \cellcolor{gray!6}{90} & \cellcolor{gray!6}{25} & \cellcolor{gray!6}{travel} & \cellcolor{gray!6}{103} & \cellcolor{gray!6}{11} & \cellcolor{gray!6}{walk} & \cellcolor{gray!6}{49} & \cellcolor{gray!6}{8} & \cellcolor{gray!6}{distance} & \cellcolor{gray!6}{3271} & \cellcolor{gray!6}{205}\\
bike & 87 & 15 & information & 65 & 21 & travel & 41 & 8 & students & 2960 & 173\\
\cellcolor{gray!6}{cycling} & \cellcolor{gray!6}{78} & \cellcolor{gray!6}{22} & \cellcolor{gray!6}{walking} & \cellcolor{gray!6}{57} & \cellcolor{gray!6}{17} & \cellcolor{gray!6}{students} & \cellcolor{gray!6}{39} & \cellcolor{gray!6}{9} & \cellcolor{gray!6}{cycling} & \cellcolor{gray!6}{2753} & \cellcolor{gray!6}{171}\\
\addlinespace
safety & 71 & 21 & walk & 53 & 13 & safety & 32 & 6 & environment & 2631 & 202\\
\cellcolor{gray!6}{health} & \cellcolor{gray!6}{65} & \cellcolor{gray!6}{21} & \cellcolor{gray!6}{weather} & \cellcolor{gray!6}{40} & \cellcolor{gray!6}{11} & \cellcolor{gray!6}{help} & \cellcolor{gray!6}{29} & \cellcolor{gray!6}{9} & \cellcolor{gray!6}{activity} & \cellcolor{gray!6}{2371} & \cellcolor{gray!6}{209}\\
physical & 63 & 18 & safety & 40 & 19 & schools & 25 & 9 & traffic & 2353 & 208\\
\cellcolor{gray!6}{traffic} & \cellcolor{gray!6}{59} & \cellcolor{gray!6}{20} & \cellcolor{gray!6}{safe} & \cellcolor{gray!6}{39} & \cellcolor{gray!6}{19} & \cellcolor{gray!6}{children} & \cellcolor{gray!6}{25} & \cellcolor{gray!6}{6} & \cellcolor{gray!6}{choice} & \cellcolor{gray!6}{2299} & \cellcolor{gray!6}{169}\\
road & 56 & 13 & services & 37 & 17 & community & 24 & 7 & physical & 2256 & 215\\
\addlinespace
\cellcolor{gray!6}{activity} & \cellcolor{gray!6}{55} & \cellcolor{gray!6}{14} & \cellcolor{gray!6}{planning} & \cellcolor{gray!6}{37} & \cellcolor{gray!6}{7} & \cellcolor{gray!6}{bus} & \cellcolor{gray!6}{18} & \cellcolor{gray!6}{4} & \cellcolor{gray!6}{trips} & \cellcolor{gray!6}{2194} & \cellcolor{gray!6}{170}\\
schools & 52 & 14 & parents & 32 & 17 & route & 17 & 5 & car & 2148 & 195\\
\cellcolor{gray!6}{children} & \cellcolor{gray!6}{47} & \cellcolor{gray!6}{15} & \cellcolor{gray!6}{sustainable} & \cellcolor{gray!6}{31} & \cellcolor{gray!6}{8} & \cellcolor{gray!6}{zone} & \cellcolor{gray!6}{16} & \cellcolor{gray!6}{6} & \cellcolor{gray!6}{safety} & \cellcolor{gray!6}{2140} & \cellcolor{gray!6}{204}\\
plan & 45 & 16 & children & 31 & 14 & resources & 16 & 6 & time & 2101 & 218\\
\cellcolor{gray!6}{students} & \cellcolor{gray!6}{44} & \cellcolor{gray!6}{14} & \cellcolor{gray!6}{child} & \cellcolor{gray!6}{31} & \cellcolor{gray!6}{12} & \cellcolor{gray!6}{day} & \cellcolor{gray!6}{16} & \cellcolor{gray!6}{4} & \cellcolor{gray!6}{factors} & \cellcolor{gray!6}{2101} & \cellcolor{gray!6}{216}\\
\addlinespace
walk & 43 & 18 & day & 29 & 13 & safe & 15 & 5 & child & 2085 & 187\\
\cellcolor{gray!6}{public} & \cellcolor{gray!6}{39} & \cellcolor{gray!6}{15} & \cellcolor{gray!6}{routes} & \cellcolor{gray!6}{28} & \cellcolor{gray!6}{14} & \cellcolor{gray!6}{planning} & \cellcolor{gray!6}{15} & \cellcolor{gray!6}{4} & \cellcolor{gray!6}{walk} & \cellcolor{gray!6}{2008} & \cellcolor{gray!6}{200}\\
community & 37 & 19 & physical & 28 & 11 & physical & 15 & 7 & public & 1983 & 208\\
\cellcolor{gray!6}{safe} & \cellcolor{gray!6}{34} & \cellcolor{gray!6}{16} & \cellcolor{gray!6}{health} & \cellcolor{gray!6}{28} & \cellcolor{gray!6}{11} & \cellcolor{gray!6}{healthy} & \cellcolor{gray!6}{14} & \cellcolor{gray!6}{6} & \cellcolor{gray!6}{age} & \cellcolor{gray!6}{1783} & \cellcolor{gray!6}{211}\\
benefits & 32 & 17 & inclement & 25 & 11 & traffic & 13 & 6 & urban & 1768 & 200\\
\addlinespace
\cellcolor{gray!6}{play} & \cellcolor{gray!6}{31} & \cellcolor{gray!6}{2} & \cellcolor{gray!6}{eligibility} & \cellcolor{gray!6}{24} & \cellcolor{gray!6}{11} & \cellcolor{gray!6}{support} & \cellcolor{gray!6}{13} & \cellcolor{gray!6}{6} & \cellcolor{gray!6}{home} & \cellcolor{gray!6}{1715} & \cellcolor{gray!6}{199}\\
resources & 30 & 13 & consortium & 24 & 9 & families & 13 & 5 & social & 1713 & 191\\
\cellcolor{gray!6}{healthy} & \cellcolor{gray!6}{29} & \cellcolor{gray!6}{16} & \cellcolor{gray!6}{region} & \cellcolor{gray!6}{23} & \cellcolor{gray!6}{10} & \cellcolor{gray!6}{way} & \cellcolor{gray!6}{12} & \cellcolor{gray!6}{5} & \cellcolor{gray!6}{different} & \cellcolor{gray!6}{1713} & \cellcolor{gray!6}{215}\\
routes & 27 & 13 & service & 22 & 11 & student & 12 & 5 & mobility & 1659 & 138\\
\cellcolor{gray!6}{lanes} & \cellcolor{gray!6}{26} & \cellcolor{gray!6}{3} & \cellcolor{gray!6}{•} & \cellcolor{gray!6}{21} & \cellcolor{gray!6}{1} & \cellcolor{gray!6}{region} & \cellcolor{gray!6}{12} & \cellcolor{gray!6}{4} & \cellcolor{gray!6}{significant} & \cellcolor{gray!6}{1650} & \cellcolor{gray!6}{208}\\
\bottomrule
\multicolumn{12}{l}{\rule{0pt}{1em}\textit{Note: }}\\
\multicolumn{12}{l}{\rule{0pt}{1em} }\\
\multicolumn{12}{l}{\rule{0pt}{1em}\textsuperscript{a} Count (n) refers to the total number of times the term is found in the corpora}\\
\multicolumn{12}{l}{\rule{0pt}{1em}\textsuperscript{b} Documents (n) refers to the total number of documents that feature the term}\\
\end{tabular}}
\end{table}

\hypertarget{bigrams-and-concordances}{%
\subsection{5.2. Bigrams and
concordances}\label{bigrams-and-concordances}}

Bigrams refer to a pair of consecutive words. We use the
\texttt{unnest\_tokens} function from the \emph{tidytext} package to
determine the bigrams. Figures \ref{fig:city-visual},
\ref{fig:consortia-visual}, and \ref{fig:school-visual} show the bigrams
that occur more than 5 times for each set of policy documents. These
figures help to make further sense of the word frequencies reported
above, and highlight the main ideas that are presented to the public in
each of the policy corpora. Municipalities primarily discuss
\emph{physical activity} (n = 53) and \emph{public health} (n = 19) in
the context of AST. In addition, \emph{travel planning} (n = 19),
\emph{bike lanes} (n = 16), and \emph{safe routes} (n = 14) are also
identified, conceivably as either proposed solutions or built
environment factors that support AST. Key issues related to transport
such as \emph{traffic safety} (n = 10), \emph{air quality} (n = 9), and
\emph{greenhouse gases} (n = 9) are conveyed to the public through these
policy documents. It is not surprising to find this focus given that
municipalities in Ontario are concerned about climate change and have
increasingly looked to active modes of travel to offset
transport-related emissions in urban areas.

Similar word bigrams are found in school board documents: \emph{travel
planning} (n = 33), \emph{safe routes} (n = 15), \emph{physical
activity} (n = 10), and \emph{public health} (n = 10) are among the most
common bigrams. Both municipalities and school boards in Ontario seem to
emphasize what can be or has been done to improve AST (i.e., policy or
planning changes), while outlining some of the benefits of AST at the
individual- or community-level to potentially encourage behaviour change
(i.e., physical activity for children or improved air quality). Unlike
other STP stakeholders, school boards also consider \emph{inclement
weather} (n = 24) and \emph{bus cancellations} (n = 13). This is likely
because many students in Ontario travel to school by bus and this
information is presented alongside AST options. Finally, transportation
consortia documents highlight topics such as \emph{physical activity} (n
= 10), \emph{pedestrian safety} (n = 8), \emph{crossing guards} (n = 6),
\emph{travel planning} (n = 6), and \emph{walk zones} (n = 6). Biking or
cycling is notably absent from transportation consortia documents.
Overall, the policy documents There appears convey an emphasis on the
built environment, rather than household decision-making.

\begin{figure}

{\centering \includegraphics[width=1\linewidth]{AST-Framing-Ontario_files/figure-latex/city-visual-1} 

}

\caption{\label{fig:city-visual}Most common bigrams found in the municipal or regional government documents.}\label{fig:city-visual}
\end{figure}

\begin{figure}

{\centering \includegraphics[width=1\linewidth]{AST-Framing-Ontario_files/figure-latex/consortia-visual-1} 

}

\caption{\label{fig:consortia-visual}Most common bigrams found in the transportation consortia documents.}\label{fig:consortia-visual}
\end{figure}

\begin{figure}

{\centering \includegraphics[width=1\linewidth]{AST-Framing-Ontario_files/figure-latex/school-visual-1} 

}

\caption{\label{fig:school-visual}Most common bigrams found in the school board documents.}\label{fig:school-visual}
\end{figure}

\begin{figure}

{\centering \includegraphics[width=1\linewidth]{AST-Framing-Ontario_files/figure-latex/policy-visual-1} 

}

\caption{\label{fig:policy-visual}Most common bigrams found across all policy documents (i.e., school board, municipality, and transportation consortia combined).}\label{fig:policy-visual}
\end{figure}

We then combine all municipality, school board, and transportation
consortia documents into one ``policy'' corpus. This enabled us to
examine and visualize the most common bigrams found across all of the
material in Ontario that was collected for this study. Figure
\ref{fig:policy-visual} shows all of the bigrams that occur more than 10
times in the policy corpus. In addition to the bigrams already
identified above, we also found \emph{mental health}, \emph{walk day},
and \emph{green communities} as common pairs of consecutive words. The
latter terms represent the significant involvement of the non-profit
organization in supporting AST initiatives through the Ontario Active
School Travel program. Overall, the policy documents from STP
stakeholder groups seem to focus on four key areas: i) benefits or
impacts of AST; ii) mechanisms of intervention; iii) concerns or
considerations; and iv) supports for AST. This interpretation indicates
that the general public accessing information about AST in Ontario is
informed about an adequate range of content related to this issue.

Next, we analyze bigrams in the academic corpus separately to make
comparisons with the policy corpus. Figure \ref{fig:academic-visual}
indicates that academic papers include several common bigrams that were
also found in the policy documents including \emph{physical activity} (n
= 1566), which is the top bigram, \emph{traffic safety} (n = 308), and
\emph{safe routes} (n = 268). However, many other factors are identified
in the research literature that are not presented to the general public
through policy documents. After \emph{physical activity}, \emph{built
environment} (n = 1175), \emph{independent mobility} (n = 774), and
\emph{urban form} (n = 352) are the most frequent pairs of consecutive
words. Academic papers also often discuss \emph{distance home} (n =
258), \emph{car ownership} (n = 254), \emph{household income} (n = 254),
and \emph{population density} (n = 205), which are factors that have
been found to influence AST. It is evident that many papers investigate
gender differences in AST given that \emph{boys girls} (n = 211) is
another common bigram. Finally, the presence of \emph{statistically
significant} among the top bigrams underscores that some researchers aim
to identify determinants using statistical measures. We find that the
academic corpus focuses on a greater range of topics than found in the
policy documents.

\begin{figure}

{\centering \includegraphics[width=1\linewidth]{AST-Framing-Ontario_files/figure-latex/academic-visual-1} 

}

\caption{\label{fig:academic-visual}Most common bigrams found in the academic papers.}\label{fig:academic-visual}
\end{figure}

We interpret the most common bigrams from the policy corpus (see Figure
\ref{fig:policy-visual}), which includes all documents from
municipalities, transportation consortia, and school boards, as the main
ideas that STP stakeholder groups are focusing on and communicating to
the public about AST. We then use the \texttt{kwic} function from the
\emph{quanteda} package to better understand the context of these key
ideas. Table \ref{tab:policy-concordance} presents some examples of the
context that was extracted from select policy documents to demonstrate
how the most common bigrams are communicated to the public.

\begin{table}

\caption{\label{tab:content-table}\label{tab:policy-concordance}The context of key terms that were identified as common bigrams.}
\centering
\begin{tabular}[t]{>{}ll>{\raggedright\arraybackslash}p{20em}}
\toprule
Terms & Stakeholder & Context\\
\midrule
\textbf{\cellcolor{gray!6}{Air Quality}} & \cellcolor{gray!6}{School Board} & \cellcolor{gray!6}{Active transportation [...] improves air quality.}\\
\textbf{Benefit} & Municipality & Stronger bones and muscles, improved self-esteem and sense of well-being while reducing stress and risk of chronic disease all benefit those who use active transportation.\\
\textbf{\cellcolor{gray!6}{Walking School Bus}} & \cellcolor{gray!6}{School Board} & \cellcolor{gray!6}{While taking part in a walking school bus, your child will enjoy seeing friends on the way to school. They will be active more often. This is also a great opportunity for your child to socialize with school friends in a monitored and safe way where they can practice social distancing, modelled by a leader.}\\
\textbf{Community} & School Board & Help your students get started on the right foot - encourage them to walk or bike to school when possible. Even leaving the car a block or two and walking the rest of the way helps. It’s good for the environment and your health, and teaches your child independence and community awareness.\\
\textbf{\cellcolor{gray!6}{Emissions}} & \cellcolor{gray!6}{Consortia} & \cellcolor{gray!6}{An active school commute also reduces congestion in school zones and contributes to reducing greenhouse gas emissions – it’s a win-win for everyone!}\\
\addlinespace
\textbf{Health} & Municipality & Active School Travel allows school-aged children the chance to participate in moderate to intense physical activity. This is linked with lower body mass index and improved cardiovascular health.\\
\textbf{\cellcolor{gray!6}{Lanes}} & \cellcolor{gray!6}{Municipality} & \cellcolor{gray!6}{We are continuing to build on the cycling and pedestrian network by adding more bike lanes, building multi-use paths and encouraging developments to provide better pedestrian/cycling environments.}\\
\textbf{Mental Health} & Municipality & ASST not only improves physical and mental health but contributes to a healthier environment and safer streets.\\
\textbf{\cellcolor{gray!6}{Physical Health}} & \cellcolor{gray!6}{Municipality} & \cellcolor{gray!6}{Encouraging Active Transportation promotes personal health and recreation, helps manage congestion, reduces emissions and supports municipal objectives for efficient land use.}\\
\bottomrule
\end{tabular}
\end{table}

\hypertarget{topic-modelling-1}{%
\subsection{5.3. Topic modelling}\label{topic-modelling-1}}

\begin{verbatim}
## Warning: `guides(<scale> = FALSE)` is deprecated. Please use `guides(<scale> =
## "none")` instead.
\end{verbatim}

\includegraphics{AST-Framing-Ontario_files/figure-latex/evaluate-lda-1.pdf}

\begin{verbatim}
## Warning: `guides(<scale> = FALSE)` is deprecated. Please use `guides(<scale> =
## "none")` instead.
\end{verbatim}

\includegraphics{AST-Framing-Ontario_files/figure-latex/evaluate-lda-2.pdf}

\begin{verbatim}
## Warning: `guides(<scale> = FALSE)` is deprecated. Please use `guides(<scale> =
## "none")` instead.
\end{verbatim}

\includegraphics{AST-Framing-Ontario_files/figure-latex/evaluate-lda-3.pdf}

\begin{verbatim}
## Warning: `guides(<scale> = FALSE)` is deprecated. Please use `guides(<scale> =
## "none")` instead.
\end{verbatim}

\includegraphics{AST-Framing-Ontario_files/figure-latex/evaluate-lda-4.pdf}

\begin{verbatim}
## Warning: `guides(<scale> = FALSE)` is deprecated. Please use `guides(<scale> =
## "none")` instead.
\end{verbatim}

\includegraphics{AST-Framing-Ontario_files/figure-latex/evaluate-lda-5.pdf}

\begin{figure}
\includegraphics[width=1\linewidth]{AST-Framing-Ontario_files/figure-latex/policy-terms-1} \caption{\label{fig:policy-terms}Topics identified in the policy corpus according to clusters of words.}\label{fig:policy-terms}
\end{figure}

\begin{figure}
\includegraphics[width=1\linewidth]{AST-Framing-Ontario_files/figure-latex/academic-terms-1} \caption{\label{fig:academic-terms}Topics identified in the academic corpus according to clusters of words.}\label{fig:academic-terms}
\end{figure}

Finally, we conduct topic modelling to examine the different topics
found in the policy and academic corpora. We focus on the policy corpus,
instead of individually assessing the municipal, school, and consortia
documents, so that we could report on the various frames used across all
documents put out by STP stakeholder groups in Ontario. To determine the
number of discrete topics in each corpus, we use the \emph{ldatuning}
package. We then use the \texttt{LDA} function from the
\emph{topicmodels} package to estimate an LDA model for each group of
documents. The parameters from the \emph{ldatuning} package suggest that
the policy corpus has between 7 and 9 topics and the academic corpus has
between 17 and 25 topics. After running the LDA model for the academic
corpus, it was too difficult to interpret a minimum of 17 topics based
on the clusters of words that were identified. We experimented with the
model by adjusting and evaluating the number of topics and found that
there were 9 distinct topics that could be interpreted, after which
there was too much overlap. Figures \ref{fig:policy-terms} and
\ref{fig:academic-terms} present the main terms that are associated with
the topics found in each corpus.

In the policy corpus, we interpret the following topics based on the
cluster of words: (1) resources for walking; (2) walking; (3) supporting
active travel; (4) bicycling and the environment; (5) benefits of active
travel; (6) safety; and (7) busing information and eligibility. These
topics indicate that STP stakeholder groups are sending the message that
walking and bicycling to school are healthy travel modes for students,
particularly as a means to get physical activity. We also found that
there is information shared to support parents and students in using
active modes to school such as the availability of cycling lanes or
route tips for walking. STP stakeholder groups are appear to be
addressing the issue of safety.

The academic corpus has a higher number of topics likely due to the
volume of papers that were sourced. The following topics were identified
based on the clusters of words: (1) built environment (2) walking
distance; (3) AST interventions; (4) parental barriers to walking; (5)
behaviours and attitudes; (6) bicycling; (7) social environment; (8)
children's independent mobility; (9) modeling trip choice; (10)
adolescents and physical activity; and (11) safe routes to school
programs. This corpus reflects a broader range of topics than the policy
corpus.

We conclude that the policy corpus primarily frames AST as a health and
environmental issue. Policy documents do not reflect the diversity of
content found in the academic papers and instead focus on a few factors.
STP stakeholder groups appear to position walking, bicycling, or rolling
to school as beneficial to individual health, as an opportunity for
physical activity or to improve mental health, and to the broader
community through a reduction in traffic and vehicle emissions. Here, we
present some examples from the policy documents to illustrate how the
health and environmental frames are communicated:

\begin{quote}
\emph{If you live in a walk zone, the best way to get to school is by
walking or biking. This promotes physical activity, helps the
environment and minimizes traffic around schools during busy times.}
(City of Barrie)
\end{quote}

\begin{quote}
\emph{Walking to school is a great way to add physical activity into
your child's busy day.} (Region of Haldimand-Norfolk Region)
\end{quote}

\begin{quote}
\emph{Active school travel is a great way for children to be physically
active, which is associated with improved physical and mental health,
while making school zones safer, by reducing traffic volumes at and
around schools.}(Region of Leeds, Grenville and Lanark)
\end{quote}

\begin{quote}
\emph{The WCDSB supports active transportation as the preferred method
of transportation to school because it is a healthy choice that has
proven links to greater student achievement.} (Waterloo Catholic
District School Board)
\end{quote}

However by examining document frequency (see Section 5.1), we found that
some terms are not present in all policy documents. This suggests that
although documents pertain to the subject of active travel or school
travel, some stakeholders across Ontario are not disseminating
information about AST. For example, a document may discuss active travel
but not to school or school travel but only by bus and not by active
modes. We manually searched the policy corpus and found that 48\% of
documents mention AST and 16\% mention STP. This confirms that many
municipalities, school boards, and transportation consortia are not
promoting AST through their webpages or indicating their involvement in
the STP process. Instead, inclement weather and impacts to busing is a
common topic addressed in school board and transportation consortia
documents.

Furthermore, we found that some policy documents make direct reference
to topics found in the academic papers. For example, some policy
documents encourage AST programs that have been researched and evaluated
in the literature, such as walking school buses, or explain how an
increase in AST will reduce parent safety concerns.

\begin{quote}
\_ For their health, safety, environment and community: Kids learn
healthy habits and concentrate better in class; One less car (yours)
reduces traffic and parking problems in school zones; Teach your kids
about traffic safety; Start a walking school bus; your kids make friends
in every grade, and that can prevent bullying.\_ (City of Guelph)
\end{quote}

\begin{quote}
\emph{There are lots of benefits in the classroom for children that walk
or cycle to school on a regular basis. Some of these benefits include
improved concentration and better coping with stress. Being outside
helps to prevent feelings of isolation and increases their social
interactions. Walking and biking to school can also save you money and
lead to fewer cars on the road.} (City of Ottawa)
\end{quote}

The secondary frame for AST in policy documents is the opportunity or
prospect of behaviour change. Some cities and schools explain how
children and parents can leave the car at home and make the journey to
school on foot or by bike by describing efforts to make AST safer. This
frame encourages the public to evaluate their own travel decisions and
to access resources (e.g., walking skills checklist) that will help them
make AST a first choice. Some documents emphasize the role of the parent
in creating opportunities for AST in their household or neighbourhood,
however few documents address the main barriers as perceived by parents
(e.g., distance, travel arrangements for multiple people, or
convenience). Examples of this secondary frame include:

\begin{quote}
\_A way to make sure your child is safe while walking to school is with
a `walking school bus.' Here are some tips for a walking school bus:
Invite families who live nearby to walk; Pick a route and take a test
walk; Take side streets and paths that are less busy with traffic;
Decide how often the group will walk together; Talk with your boss to
adjust your day; Have fun! (City of Ottawa)
\end{quote}

\begin{quote}
Help your students get started on the right foot - encourage them to
walk or bike to school when possible. Even leaving the car a block or
two and walking the rest of the way helps. It's good for the environment
and your health, and teaches your child independence and community
awareness.\_ (Halton District School Board)
\end{quote}

\begin{quote}
\emph{Want to boost your child's mental and physical health? Ottawa
Public Health, City of Ottawa, and OSTA have produced a tipsheet for
parents about ``active transportation'' to school -- fitting walking and
wheeling into your daily routine.} (Ottawa-Carleton District School
Board)
\end{quote}

Finally, the policy documents discuss proposed solutions to encourage
AST. STP stakeholder groups seem to be communicating that AST is
possible and safe as a result of improvements to the built environment
and available resources for parents and children. A final focus in the
policy documents is on various efforts that are underway to support AST
including route planning. A few examples include:

\begin{quote}
\emph{School Travel Planning is a community-based approach that aims to
increase the number of students and adults choosing active and
sustainable travel to get to and from school. This approach addresses
concerns about safety, physical activity, and the environment.} (City of
Hamilton)
\end{quote}

\begin{quote}
\emph{Today, as more and more of our neighbourhoods are being
retrofitted with new sidewalks and bike lanes, pedestrian crossovers,
street lights, reduced speed limits and/or crossing guards, the walk or
bike ride to and from school has never been easier, safer or healthier.}
(Hamilton-Wentworth District Catholic School Board)
\end{quote}

\hypertarget{discussion}{%
\section{6. Discussion}\label{discussion}}

\hypertarget{framing-in-stp-documents}{%
\subsection{6.1. Framing in STP
documents}\label{framing-in-stp-documents}}

Using topic modelling, we analyzed how AST is framed in documents
available to the general public on the websites of STP stakeholder
groups in Ontario. We found that AST is primarily framed to parents as
beneficial to the health and wellbeing of children and to environmental
sustainability. This was confirmed by the most common bigrams identified
in the policy corpus, as well as the topics identified by the LDA model.
The policy documents adequately reflect the evidence that AST
contributes positively to children's physical health (see Faulkner et
al., 2009; Schoeppe et al., 2015), although the statements regarding the
benefits of AST to children's school performance and cognitive
development need to be studied further to produce stronger evidence
({\textbf{???}}). STP stakeholder groups also communicate that
increasing AST may reduce traffic near and around schools. This
presumably is communicated to alleviate parental concerns about traffic
and safety (Evers et al., 2014; Mammen et al., 2012; Rothman et al.,
2015; Wilson et al., 2018) or reduce the frequency of risky behaviours
from drivers around schools (Rothman et al., 2017). However, the
documents do not explain why declining rates of AST are a problem for
the particular community or convey any urgency to this issue so that it
attracts the attention of parents, the general public, or policy makers.
Communicating the potential outcomes of increased AST may be persuasive
arguments to motivate behaviour change, but these documents do not
appear to encourage parents or the general public to view their
behaviour as problematic or unhealthy for their children's development.
For example, Canadian parents who escort their children to school have
reported concerns about traffic volume around schools (Mammen et al.,
2012), but may not recognize that their own behaviour contributes to the
problem that is perceived to prevent their child from safely walking or
bicycling to school (Collins and Kearns, 2001).

The secondary frame of AST that we interpreted based on the LDA model is
the opportunity or prospect of behaviour change. STP stakeholder groups
identified different ways that parents could encourage or support their
child(ren) to commute to school by using active modes. Some school
boards and municipalities also shared resources such as walking tip
sheets and guidance for starting a walking school bus. Other advice
included dropping children off one or two blocks away from school so
that they could walk or bike part of the trip. The general emphasis is
communicating information that could change parental perceptions about
the ease of their children using active modes to school, which may be
seen by STP stakeholder groups as a ``modifiable'' factor (see Riazi et
al., 2019). In turn, this could encourage parents to modify their
routines and incorporate opportunities for their children to use active
modes to school.

We found that interpersonal behaviour changes achieved through
educational strategies and engineering changes (e.g., traffic calming
measures, reduced speed limits, etc.) are communicated to the general
public as potential solutions to address the decline of AST. This
reflects findings from the AST literature (Panter et al., 2010), but
these solutions have to sufficiently address real and perceived barriers
from parents about the household environment and the built environment
(see Section 2.2). In most documents, STP stakeholder groups provide
parents with resources that help them switch modes for the school
commute and/or on reassure them that the built environment has become
safer and friendlier to active travelers as a result of infrastructure
or policy improvements. As previously noted, the documents do not appear
to present a strong ``call to action'' that urges behaviour change for
parents or the general public, which may affect the success of STP
interventions. On the other hand, the emphasis on engineering changes
may reflect the strong engagement of the ``policy stream'' (Kingdon,
1984) and the feasible options to address the problem, since engineering
staff and municipal representatives are common STP stakeholders
(Buttazzoni et al., 2018; Mammen et al., 2015).

Given the range of factors that influence AST in a Canadian setting
(\emph{inter alia}, see Mammen et al., 2012; Mitra, 2013; Rothman et
al., 2018; Wilson et al., 2018), STP stakeholder groups must decide
which points of intervention that they can reasonably have some
influence towards. This would also depend on the local context to ensure
that barriers to AST as perceived by parents are addressed. It could be
that STP stakeholder groups perceive to have more control over
micro-scale elements of the built environment, like traffic calming
measures or speed limits, rather ``non-modifiable'' factors like street
density or land use mix. It would certainly be much more difficult for
STP stakeholder groups to influence the number of trips to different
destinations that households need to make or to intervene in residential
location, which affects distance to school. A focus on changing parental
perceptions is recommended by many scholars since parents are the
``gatekeeper'' for children's mobility, but the information communicated
by STP stakeholder groups in Ontario should reflect a broader range of
factors that influence AST.

\hypertarget{implications-for-school-travel-planning}{%
\subsection{6.2. Implications for school travel
planning}\label{implications-for-school-travel-planning}}

There is also a noticeable lack of focus on individual and interpersonal
determinants of AST in the policy documents. For example, the role of
convenience and inclement weather in shaping household travel decisions
(Buliung et al., 2011) or the age at which children become allowed to
travel by active modes or independently (Mammen et al., 2012) were not
discussed. The desire to escort children to school, which has been noted
by parents as a reason to continue driving (Westman, Friman, et al.,
2017), is also not adequately addressed by STP stakeholder groups. The
complexity of travel arrangements that must be coordinated by households
(see Buliung et al., 2021) and the dominant parenting culture to protect
children from risk ({\textbf{???}}) are likewise overlooked. STP
stakeholders may wish to emphasize to parents that older children have
greater traffic safety and cognitive abilities to navigate their own
routes to school, which could encourage parents to develop their
child(ren)'s travel skills over time. Additional attention should be
paid to the reasons why parents perceive driving to be more convenient.

Coming back to Kingdon's Multiple Streams Framework (1984), we posited
that STP stakeholder groups need to raise awareness about the AST issue
among parents and the general public through the problem stream. This
could influence perceptions about the issue and encourage behaviour
change. Based on the policy documents, it is not obvious that AST has
declined significantly over the past few decades in Canada or that this
issue merits urgent attention. Many municipalities, schools, and
transportation consortia in Ontario use similar content in their public
documents about AST. It would be worthwhile for STP stakeholder groups
to frame the decline of AST as a ``problem'' worth addressing with
information specific to each local context. STP stakeholders also need
to ensure that their communication accounts for the ways in which
Canadian parents perceive the built environment in different ways than
planners (see Buliung et al., 2021). If there is ``conceptual
asymmetry'' as Buliung et al.~(2021) identified, the materials
disseminated by STP stakeholders may not resonate or reach parents.
Future research or STP activities should investigate how parents or the
general public respond to messages or information that encourages the
adoption of AST and evaluate how effective they are to motivate
behaviour change. It would be helpful to know which frames would most
encourage behaviour change or increase political support for
interventions that address barriers to AST. This type of information
could ensure that educational strategies and promotional materials
increase buy-in for their target audience.

Although the ``policy stream'' appears to have been adequately engaged
in Ontario communities, it could be that the decline of AST has not been
sufficiently framed as ``problematic'' to warrant more widespread
awareness or solutions. If Canadian STP stakeholder groups wish to
involve more local residents in their efforts (Buttazzoni et al., 2018),
it would also be worthwhile for them to produce different material that
communicates why this issue is important to the general public,
regardless of whether they currently have children commuting to/from
school. The ``political steam'' could also be further engaged by working
with advocacy groups who could attract more support from a broader
coalition of individuals.

Finally, Rothman et al.~(2018) note that local solutions to address
specific school challenges are needed, and we extend this recommendation
to include local context in STP messaging about AST. Instead of generic
content about AST, stakeholders involved in STP should produce more
information that specifically responds to parents' perceived barriers
and challenges that were identified in that local context. For example,
STP materials could explain how known concerns in a particular
municipality or school area have been addressed through local solutions.
This could reflect that solutions have been identified based on what was
heard from the community.

\hypertarget{limitations}{%
\subsection{6.3. Limitations}\label{limitations}}

A limitation of this study is that we only analyzed texts that were
easily accessible to the general public on the websites of STP
stakeholder groups in Ontario. Parents likely receive information about
AST directly from schools, which may contain more content that reflects
the local barriers to AST.

\hypertarget{conclusion}{%
\section{7. Conclusion}\label{conclusion}}

We used text mining and topic modelling to examine how different school
travel planning stakeholder (STP) groups in Ontario frame the issue of
AST. STP stakeholder groups frame AST as a health and environmental
issue that can be addressed through household behaviour change and
engineering solutions. Policy documents reveal that STP stakeholder
groups are focusing on ``modifiable factors'' such as parental
perceptions or micro-scale elements in the built environment to increase
rates of AST. However, AST may not be framed sufficiently as a
``problem'' that requires urgent intervention, which may impact how
parents respond to behaviour change initiatives and limit awareness in
the general public. In their public materials about AST, STP stakeholder
groups should present solutions for a broader range of factors that
influence AST and emphasize why AST rates should increase in local
communities.

\hypertarget{references}{%
\section*{References}\label{references}}
\addcontentsline{toc}{section}{References}

\hypertarget{refs}{}
\leavevmode\hypertarget{ref-albalawiUsingTopicModeling2020}{}%
Albalawi, R., Yeap, T.H., Benyoucef, M., 2020. Using Topic Modeling
Methods for Short-Text Data: A Comparative Analysis. Frontiers in
Artificial Intelligence 3, 42.
doi:\href{https://doi.org/10.3389/frai.2020.00042}{10.3389/frai.2020.00042}

\leavevmode\hypertarget{ref-badlandDevelopmentSystemsModel2016}{}%
Badland, H., Kearns, R., Carroll, P., Oliver, M., Mavoa, S., Donovan,
P., Parker, K., Chaudhury, M., Lin, E.-Y., Witten, K., 2016. Development
of a systems model to visualise the complexity of children's independent
mobility. Children's Geographies 14, 91--100.
doi:\href{https://doi.org/10.1080/14733285.2015.1021240}{10.1080/14733285.2015.1021240}

\leavevmode\hypertarget{ref-buliungSchoolTravelPlanning2011}{}%
Buliung, R., Faulkner, G., Beesley, T., Kennedy, J., 2011. School Travel
Planning: Mobilizing School and Community Resources to Encourage Active
School Transportation. Journal of School Health 81, 704--712.
doi:\href{https://doi.org/10.1111/j.1746-1561.2011.00647.x}{10.1111/j.1746-1561.2011.00647.x}

\leavevmode\hypertarget{ref-buliungLivingJourneySchool2021}{}%
Buliung, R., Hess, P., Flowers, L., Moola, F.J., Faulkner, G., 2021.
Living the journey to school: Conceptual asymmetry between parents and
planners on the journey to school. Social Science \& Medicine 284,
114237.
doi:\href{https://doi.org/10.1016/j.socscimed.2021.114237}{10.1016/j.socscimed.2021.114237}

\leavevmode\hypertarget{ref-buttazzoniPromotingActiveSchool2019}{}%
Buttazzoni, A.N., Clark, A.F., Seabrook, J.A., Gilliland, J.A., 2019.
Promoting active school travel in elementary schools: A regional case
study of the school travel planning intervention. Journal of Transport
\& Health 12, 206--219.
doi:\href{https://doi.org/10.1016/j.jth.2019.01.007}{10.1016/j.jth.2019.01.007}

\leavevmode\hypertarget{ref-buttazzoniSupportingActiveSchool2018}{}%
Buttazzoni, A.N., Coen, S.E., Gilliland, J.A., 2018. Supporting active
school travel: A qualitative analysis of implementing a regional safe
routes to school program. Social Science \& Medicine 212, 181--190.
doi:\href{https://doi.org/10.1016/j.socscimed.2018.07.032}{10.1016/j.socscimed.2018.07.032}

\leavevmode\hypertarget{ref-borrestadExperiencesRandomisedControlled2012}{}%
Børrestad, L.A.B., Østergaard, L., Andersen, L.B., Bere, E., 2012.
Experiences from a randomised, controlled trial on cycling to school:
Does cycling increase cardiorespiratory fitness? Scandinavian Journal of
Public Health 40, 245--252.
doi:\href{https://doi.org/10.1177/1403494812443606}{10.1177/1403494812443606}

\leavevmode\hypertarget{ref-GreenCommunities2016}{}%
Canada, G.C., 2020a. Ontario active school travel fund {[}WWW
Document{]}. URL
\url{https://ontarioactiveschooltravel.ca/ontario-active-school-travel-fund/}

\leavevmode\hypertarget{ref-GreenCommunitiesProjects}{}%
Canada, G.C., 2020b. OAST fund project directory {[}WWW Document{]}. URL
\url{https://ontarioactiveschooltravel.ca/wp-content/uploads/2020/10/OAST-Fund-Matrix-Directory.pdf}

\leavevmode\hypertarget{ref-chenPromotingActiveStudent2018}{}%
Chen, P., Jiao, J., Xu, M., Gao, X., Bischak, C., 2018. Promoting active
student travel: A longitudinal study. Journal of Transport Geography 70,
265--274.
doi:\href{https://doi.org/10.1016/j.jtrangeo.2018.06.015}{10.1016/j.jtrangeo.2018.06.015}

\leavevmode\hypertarget{ref-collinsSafeJourneysEnterprising2001}{}%
Collins, D.C.A., Kearns, R.A., 2001. The safe journeys of an
enterprising school: Negotiating landscapes of opportunity and risk.
Health \& Place 7, 293--306.
doi:\href{https://doi.org/10.1016/S1353-8292(01)00021-1}{10.1016/S1353-8292(01)00021-1}

\leavevmode\hypertarget{ref-demeesterParentalPerceivedNeighborhood2014}{}%
De Meester, F., Van Dyck, D., De Bourdeaudhuij, I., Cardon, G., 2014.
Parental perceived neighborhood attributes: Associations with active
transport and physical activity among 10-12 year old children and the
mediating role of independent mobility. BMC public health 14, 631.
doi:\href{https://doi.org/10.1186/1471-2458-14-631}{10.1186/1471-2458-14-631}

\leavevmode\hypertarget{ref-depouxCommunicatingClimateChange2017}{}%
Depoux, A., Hémono, M., Puig-Malet, S., Pédron, R., Flahault, A., 2017.
Communicating climate change and health in the media. Public Health
Reviews 38, 7.
doi:\href{https://doi.org/10.1186/s40985-016-0044-1}{10.1186/s40985-016-0044-1}

\leavevmode\hypertarget{ref-eversParentSafetyPerceptions2014}{}%
Evers, C., Boles, S., Johnson-Shelton, D., Schlossberg, M., Richey, D.,
2014. Parent safety perceptions of child walking routes. Journal of
Transport \& Health 1, 108--115.
doi:\href{https://doi.org/10.1016/j.jth.2014.03.003}{10.1016/j.jth.2014.03.003}

\leavevmode\hypertarget{ref-faulknerActiveSchoolTransport2009}{}%
Faulkner, G.E.J., Buliung, R.N., Flora, P.K., Fusco, C., 2009. Active
school transport, physical activity levels and body weight of children
and youth: A systematic review. Preventive Medicine 48, 3--8.
doi:\href{https://doi.org/10.1016/j.ypmed.2008.10.017}{10.1016/j.ypmed.2008.10.017}

\leavevmode\hypertarget{ref-faulknerSchoolTravelChildren2013}{}%
Faulkner, G., Stone, M., Buliung, R., Wong, B., Mitra, R., 2013. School
travel and children's physical activity: A cross-sectional study
examining the influence of distance. BMC public health 13, 1166.
doi:\href{https://doi.org/10.1186/1471-2458-13-1166}{10.1186/1471-2458-13-1166}

\leavevmode\hypertarget{ref-fuscoUnderstandingChildrenPerceptions2012}{}%
Fusco, C., Moola, F., Faulkner, G., Buliung, R., Richichi, V., 2012.
Toward an understanding of children's perceptions of their transport
geographies: (Non)Active school travel and visual representations of the
built environment. Journal of Transport Geography, Special Section On
Child \& Youth Mobility 20, 62--70.
doi:\href{https://doi.org/10.1016/j.jtrangeo.2011.07.001}{10.1016/j.jtrangeo.2011.07.001}

\leavevmode\hypertarget{ref-gamsonChangingCulture1987}{}%
Gamson, W.A., Modigliani, A., 1987. The changing culture of affirmative
action. JAI Press.

\leavevmode\hypertarget{ref-goffmanFrameAnalysisEssay1974}{}%
Goffman, E., 1974. Frame analysis: An essay on the organization of
experience, Frame analysis: An essay on the organization of experience.
Harvard University Press, Cambridge, MA, US.

\leavevmode\hypertarget{ref-ikedaAssociationsChildrenActive2018}{}%
Ikeda, E., Hinckson, E., Witten, K., Smith, M., 2018. Associations of
children's active school travel with perceptions of the physical
environment and characteristics of the social environment: A systematic
review. Health \& Place 54, 118--131.
doi:\href{https://doi.org/10.1016/j.healthplace.2018.09.009}{10.1016/j.healthplace.2018.09.009}

\leavevmode\hypertarget{ref-jacobiQuantitativeAnalysisLarge2016}{}%
Jacobi, C., van Atteveldt, W., Welbers, K., 2016. Quantitative analysis
of large amounts of journalistic texts using topic modelling. Digital
Journalism 4, 89--106.
doi:\href{https://doi.org/10.1080/21670811.2015.1093271}{10.1080/21670811.2015.1093271}

\leavevmode\hypertarget{ref-jensenHowObtainHealthy2008}{}%
Jensen, S.U., 2008. How to obtain a healthy journey to school.
Transportation Research Part A: Policy and Practice 42, 475--486.
doi:\href{https://doi.org/10.1016/j.tra.2007.12.001}{10.1016/j.tra.2007.12.001}

\leavevmode\hypertarget{ref-kingdonAgendasAlternativesPublic1984}{}%
Kingdon, J.W. (Ed.), 1984. Agendas, alternatives and public policies.
Pearson.

\leavevmode\hypertarget{ref-langUnderstandingModalChoice2011}{}%
Lang, D., Collins, D., Kearns, R., 2011. Understanding modal choice for
the trip to school. Journal of Transport Geography 19, 509--514.
doi:\href{https://doi.org/10.1016/j.jtrangeo.2010.05.005}{10.1016/j.jtrangeo.2010.05.005}

\leavevmode\hypertarget{ref-laroucheRatherBikeSchool2016}{}%
Larouche, R., Stone, M., Buliung, R.N., Faulkner, G., 2016. ''I'd rather
bike to school!'': Profiling children who would prefer to cycle to
school. Journal of Transport \& Health 3, 377--385.
doi:\href{https://doi.org/10.1016/j.jth.2016.06.010}{10.1016/j.jth.2016.06.010}

\leavevmode\hypertarget{ref-larsenRouteBasedAnalysisCapture2012}{}%
Larsen, K., Gilliland, J., Hess, P.M., 2012. Route-Based Analysis to
Capture the Environmental Influences on a Child's Mode of Travel between
Home and School. Annals of the Association of American Geographers 102,
1348--1365.
doi:\href{https://doi.org/10.1080/00045608.2011.627059}{10.1080/00045608.2011.627059}

\leavevmode\hypertarget{ref-mahFramecriticalPolicyAnalysis2014}{}%
Mah, C.L., Hamill, C., Rondeau, K., McIntyre, L., 2014. A frame-critical
policy analysis of Canada's response to the World Food Summit 19982008.
Archives of Public Health 72, 41.
doi:\href{https://doi.org/10.1186/2049-3258-72-41}{10.1186/2049-3258-72-41}

\leavevmode\hypertarget{ref-maibachReframingClimateChange2010}{}%
Maibach, E.W., Nisbet, M., Baldwin, P., Akerlof, K., Diao, G., 2010.
Reframing climate change as a public health issue: An exploratory study
of public reactions. BMC Public Health 10, 1--11.
doi:\href{https://doi.org/10.1186/1471-2458-10-299}{10.1186/1471-2458-10-299}

\leavevmode\hypertarget{ref-mammenUnderstandingDriveEscort2012}{}%
Mammen, G., Faulkner, G., Buliung, R., Lay, J., 2012. Understanding the
drive to escort: A cross-sectional analysis examining parental attitudes
towards children's school travel and independent mobility. BMC public
health 12, 862.
doi:\href{https://doi.org/10.1186/1471-2458-12-862}{10.1186/1471-2458-12-862}

\leavevmode\hypertarget{ref-mammenSchoolTravelPlanning2014}{}%
Mammen, G., Stone, M.R., Buliung, R., Faulkner, G., 2014. School travel
planning in Canada: Identifying child, family, and school-level
characteristics associated with travel mode shift from driving to active
school travel. Journal of Transport \& Health, Walking \& Cycling: The
contributions of health and transport geography 1, 288--294.
doi:\href{https://doi.org/10.1016/j.jth.2014.09.004}{10.1016/j.jth.2014.09.004}

\leavevmode\hypertarget{ref-mammenPuttingSchoolTravel2015}{}%
Mammen, G., Stone, M.R., Buliung, R., Faulkner, G., 2015. ''Putting
school travel on the map'': Facilitators and barriers to implementing
school travel planning in Canada. Journal of Transport \& Health 2,
318--326.
doi:\href{https://doi.org/10.1016/j.jth.2015.05.003}{10.1016/j.jth.2015.05.003}

\leavevmode\hypertarget{ref-mammenActiveSchoolTravel2014}{}%
Mammen, G., Stone, M.R., Faulkner, G., Ramanathan, S., Buliung, R.,
O'Brien, C., Kennedy, J., 2014. Active school travel: An evaluation of
the Canadian school travel planning intervention. Preventive Medicine
60, 55--59.
doi:\href{https://doi.org/10.1016/j.ypmed.2013.12.008}{10.1016/j.ypmed.2013.12.008}

\leavevmode\hypertarget{ref-michailChildrenExperiencesTheir2021}{}%
Michail, N., Ozbil, A., Parnell, R., Wilkie, S., 2021. Children's
Experiences of Their Journey to School: Integrating Behaviour Change
Frameworks to Inform the Role of the Built Environment in Active School
Travel Promotion. International Journal of Environmental Research and
Public Health 18, 4992.
doi:\href{https://doi.org/10.3390/ijerph18094992}{10.3390/ijerph18094992}

\leavevmode\hypertarget{ref-mitraIndependentMobilityMode2013}{}%
Mitra, R., 2013. Independent Mobility and Mode Choice for School
Transportation: A Review and Framework for Future Research. Transport
Reviews 33, 21--43.
doi:\href{https://doi.org/10.1080/01441647.2012.743490}{10.1080/01441647.2012.743490}

\leavevmode\hypertarget{ref-panFramingAnalysisApproach1993}{}%
Pan, Z., Kosicki, G.M., 1993. Framing analysis: An approach to news
discourse. Political Communication 10, 55--75.
doi:\href{https://doi.org/10.1080/10584609.1993.9962963}{10.1080/10584609.1993.9962963}

\leavevmode\hypertarget{ref-panterAttitudesSocialSupport2010}{}%
Panter, J.R., Jones, A.P., Sluijs, E.M.F. van, Griffin, S.J., 2010.
Attitudes, social support and environmental perceptions as predictors of
active commuting behaviour in school children. Journal of Epidemiology
\& Community Health 64, 41--48.
doi:\href{https://doi.org/10.1136/jech.2009.086918}{10.1136/jech.2009.086918}

\leavevmode\hypertarget{ref-pontEnvironmentalCorrelatesChildren2009}{}%
Pont, K., Ziviani, J., Wadley, D., Bennett, S., Abbott, R., 2009.
Environmental correlates of children's active transportation: A
systematic literature review. Health \& Place 15, 849--862.
doi:\href{https://doi.org/10.1016/j.healthplace.2009.02.002}{10.1016/j.healthplace.2009.02.002}

\leavevmode\hypertarget{ref-reynardGrowthResilienceHow2021}{}%
Reynard, D., Collins, D., Shirgaokar, M., 2021. Growth over resilience:
How Canadian municipalities frame the challenge of reducing carbon
emissions. Local Environment 0, 1--13.
doi:\href{https://doi.org/10.1080/13549839.2021.1892046}{10.1080/13549839.2021.1892046}

\leavevmode\hypertarget{ref-riaziCorrelatesChildrenIndependent2019}{}%
Riazi, N.A., Blanchette, S., Trudeau, F., Larouche, R., Tremblay, M.S.,
Faulkner, G., 2019. Correlates of Children's Independent Mobility in
Canada: A Multi-Site Study. International Journal of Environmental
Research and Public Health 16.
doi:\href{https://doi.org/10.3390/ijerph16162862}{10.3390/ijerph16162862}

\leavevmode\hypertarget{ref-romeroChildrenExperiencesEnjoyment2015}{}%
Romero, V., 2015. Children׳s experiences: Enjoyment and fun as
additional encouragement for walking to school. Journal of Transport \&
Health 2, 230--237.
doi:\href{https://doi.org/10.1016/j.jth.2015.01.002}{10.1016/j.jth.2015.01.002}

\leavevmode\hypertarget{ref-rothmanSchoolEnvironmentStudent2017}{}%
Rothman, L., Buliung, R., Howard, A., Macarthur, C., Macpherson, A.,
2017. The school environment and student car drop-off at elementary
schools. Travel Behaviour and Society 9, 50--57.
doi:\href{https://doi.org/10.1016/j.tbs.2017.03.001}{10.1016/j.tbs.2017.03.001}

\leavevmode\hypertarget{ref-rothmanAssociationsParentsPerception2015}{}%
Rothman, L., Buliung, R., To, T., Macarthur, C., Macpherson, A., Howard,
A., 2015. Associations between parents׳ perception of traffic danger,
the built environment and walking to school. Journal of Transport \&
Health 2, 327--335.
doi:\href{https://doi.org/10.1016/j.jth.2015.05.004}{10.1016/j.jth.2015.05.004}

\leavevmode\hypertarget{ref-rothmanActiveSchoolTransportation2021}{}%
Rothman, L., Hagel, B., Howard, A., Cloutier, M.S., Macpherson, A.,
Aguirre, A.N., McCormack, G.R., Fuselli, P., Buliung, R., HubkaRao, T.,
Ling, R., Zanotto, M., Rancourt, M., Winters, M., 2021. Active school
transportation and the built environment across Canadian cities:
Findings from the child active transportation safety and the environment
(CHASE) study. Preventive Medicine 146, 106470.
doi:\href{https://doi.org/10.1016/j.ypmed.2021.106470}{10.1016/j.ypmed.2021.106470}

\leavevmode\hypertarget{ref-rothmanDeclineActiveSchool2018}{}%
Rothman, L., Macpherson, A.K., Ross, T., Buliung, R.N., 2018. The
decline in active school transportation (AST): A systematic review of
the factors related to AST and changes in school transport over time in
North America. Preventive Medicine 111, 314--322.
doi:\href{https://doi.org/10.1016/j.ypmed.2017.11.018}{10.1016/j.ypmed.2017.11.018}

\leavevmode\hypertarget{ref-schoeppeAssociationsChildrenActive2015}{}%
Schoeppe, S., Duncan, M.J., Badland, H.M., Oliver, M., Browne, M., 2015.
Associations between children׳s active travel and levels of physical
activity and sedentary behavior. Journal of Transport \& Health 2,
336--342.
doi:\href{https://doi.org/10.1016/j.jth.2015.05.001}{10.1016/j.jth.2015.05.001}

\leavevmode\hypertarget{ref-silgeTextMining2021}{}%
Silge, J., Robinson, D., 2021. Text Mining with R.

\leavevmode\hypertarget{ref-starkExploringIndependentActive2018}{}%
Stark, J., Frühwirth, J., Aschauer, F., 2018. Exploring independent and
active mobility in primary school children in Vienna. Journal of
Transport Geography 68, 31--41.
doi:\href{https://doi.org/10.1016/j.jtrangeo.2018.02.007}{10.1016/j.jtrangeo.2018.02.007}

\leavevmode\hypertarget{ref-vandenbergFactorsAffectingParental2020}{}%
van den Berg, P., Waygood, E.O.D., van de Craats, I., Kemperman, A.,
2020. Factors affecting parental safety perception, satisfaction with
school travel and mood in primary school children in the Netherlands.
Journal of Transport \& Health 16, 100837.
doi:\href{https://doi.org/10.1016/j.jth.2020.100837}{10.1016/j.jth.2020.100837}

\leavevmode\hypertarget{ref-vangoeverdenSchoolTravelBehaviour2013}{}%
van Goeverden, C.D., de Boer, E., 2013. School travel behaviour in the
Netherlands and Flanders. Transport Policy, ''Understanding behavioural
change: An international perspective on sustainable travel behaviours
and their motivations'': Selected Papers from the 12th World Conference
on Transportation Research 26, 73--84.
doi:\href{https://doi.org/10.1016/j.tranpol.2013.01.004}{10.1016/j.tranpol.2013.01.004}

\leavevmode\hypertarget{ref-waygoodChildrenTravelIncidental2015}{}%
Waygood, E.O.D., Friman, M., 2015. Children's travel and incidental
community connections. Travel Behaviour and Society 2, 174--181.
doi:\href{https://doi.org/10.1016/j.tbs.2015.03.003}{10.1016/j.tbs.2015.03.003}

\leavevmode\hypertarget{ref-waygoodTransportChildrenWellbeing2020}{}%
Waygood, E.O.D., Friman, M., Olsson, L.E., Mitra, R. (Eds.), 2020.
Transport and Children's Wellbeing, in: Transport and Children's
Wellbeing. Elsevier, pp. i--ii.
doi:\href{https://doi.org/10.1016/B978-0-12-814694-1.09993-0}{10.1016/B978-0-12-814694-1.09993-0}

\leavevmode\hypertarget{ref-waygoodJapanMaintainingHigh2020}{}%
Waygood, E.O.D., Taniguchi, A., 2020. Japan: Maintaining high levels of
walking, in: Waygood, E.O.D., Friman, M., Olsson, L.E., Mitra, R.
(Eds.), Transport and Children's Wellbeing. Elsevier, pp. 297--316.
doi:\href{https://doi.org/10.1016/B978-0-12-814694-1.00016-6}{10.1016/B978-0-12-814694-1.00016-6}

\leavevmode\hypertarget{ref-weathersDevelopmentsFramingClimate2016}{}%
Weathers, M.R., Kendall, B.E., 2016. Developments in the Framing of
Climate Change as a Public Health Issue in US Newspapers. Environmental
Communication 10, 593--611.
doi:\href{https://doi.org/10.1080/17524032.2015.1050436}{10.1080/17524032.2015.1050436}

\leavevmode\hypertarget{ref-westmanWhatDrivesThem2017}{}%
Westman, J., Friman, M., Olsson, L.E., 2017. What Drives Them to
Drive?Parents' Reasons for Choosing the Car to Take Their Children to
School. Frontiers in Psychology 8.
doi:\href{https://doi.org/10.3389/fpsyg.2017.01970}{10.3389/fpsyg.2017.01970}

\leavevmode\hypertarget{ref-westmanChildrenTravelSchool2017}{}%
Westman, J., Olsson, L.E., Gärling, T., Friman, M., 2017. Children's
travel to school: Satisfaction, current mood, and cognitive performance.
Transportation 44, 1365--1382.
doi:\href{https://doi.org/10.1007/s11116-016-9705-7}{10.1007/s11116-016-9705-7}

\leavevmode\hypertarget{ref-wilsonUnderstandingChildParent2018}{}%
Wilson, K., Clark, A.F., Gilliland, J.A., 2018. Understanding child and
parent perceptions of barriers influencing children's active school
travel. BMC public health 18, 1053.
doi:\href{https://doi.org/10.1186/s12889-018-5874-y}{10.1186/s12889-018-5874-y}

\leavevmode\hypertarget{ref-zwertsHowChildrenView2010}{}%
Zwerts, E., Allaert, G., Janssens, D., Wets, G., Witlox, F., 2010. How
children view their travel behaviour: A case study from Flanders
(Belgium). Journal of Transport Geography 18, 702--710.
doi:\href{https://doi.org/10.1016/j.jtrangeo.2009.10.002}{10.1016/j.jtrangeo.2009.10.002}


\end{document}

